\documentclass[a4paper, 12pt]{article}
\usepackage[utf8]{inputenc}
\usepackage[T1, T2A]{fontenc}
\usepackage[a4paper, top=2cm, bottom=2cm, left=1cm, right=1cm, marginparwidth=1.75cm]{geometry}
\usepackage{graphicx}
\usepackage{amsmath}
\usepackage{indentfirst}
\usepackage[english, russian]{babel}
\usepackage[section,above,below]{placeins}
\usepackage[noend]{algorithmic}
\usepackage{amssymb}
\usepackage{amsfonts}
\usepackage{pdfpages} 

\newcommand{\ww}{\omega}
\newcommand{\dw}{\Delta \omega}

\begin{document}
    
\begin{equation}
    \int_{\ww_j-\dw}^{\ww_j} \left(1+ \frac{\ww-\ww_j}{\dw} \right) e^{- i \ww t}d\ww=
\frac{i}{t} e^{- i \ww_j t}\left(1-\frac{i}{t\dw}\left( 1-e^{i \dw t} \right)  \right) 
\end{equation}

\begin{equation}
    \int_{\ww_j}^{\ww_j+\dw} \left(1- \frac{\ww-\ww_j}{\dw} \right) e^{- i \ww t}d\ww=
\frac{i}{t} e^{- i \ww_j t}\left(-1-\frac{i}{t\dw}\left( 1-e^{-i \dw t} \right)  \right)
\end{equation}

\begin{equation}
    f_c=\frac{\ww_c}{2 \pi}
\end{equation}
\begin{equation}
    T=\frac{1}{f_c}=\frac{2 \pi}{\ww_c}
\end{equation}

\begin{equation}
    V_1(t)=\sin (2 \pi f_c t) \cdot \sin(\pi f_c \frac{t}{2})=\sin (\ww_c t) \cdot \sin(\ww_c \frac{t}{4})
\end{equation}

Для функции $f_1$ явные формулы:
\begin{equation}
    f_1(\ww)=i(e_1-e_2-e_3+e_4),
\end{equation}
\begin{equation}
    e_i=\dfrac{e^{\frac{i \pi}{\ww_c} \ww_i} -1}{\ww_i}
\end{equation}
\begin{equation}
    \ww_1=5\ww_c+4 \ww,\ww_2=3\ww_c+4 \ww,\ww_3=-3\ww_c+4 \ww,\ww_4=-5\ww_c+4 \ww
\end{equation}
Аналогично
\begin{equation}
    f_2(\ww)=s\left(\dfrac{2 \pi N}{\ww_c}\right) - s(0)
\end{equation}
\begin{equation}
    s(t)=\frac{1}{8} \left(e_1+e_2-e_3-e_4-2(e_5-e_6)    \right)
\end{equation}
\begin{equation}
    e_i=\dfrac{e^{i t \ww_i}}{\ww_i}
\end{equation}
\begin{multline}
    \ww_1=\ww_c \dfrac{N+1}{N}+\ww,\\
    \ww_2=\ww_c \dfrac{N-1}{N}+\ww,\\
    \ww_3=\ww_c \dfrac{1-N}{N}+\ww,\\
    \ww_4=-\ww_c\dfrac{N+1}{N}+\ww,\\
    \ww_5=\ww_c +\ww,\\
    \ww_6=-\ww_c +\ww
\end{multline}


\end{document}